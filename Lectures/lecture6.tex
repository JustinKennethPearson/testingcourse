\documentclass{beamer}
\usetheme{default}
\usepackage{tikz}
\usetikzlibrary{positioning}
\setbeamertemplate{footline}[page number]
\setbeamertemplate{navigation symbols}{}
\usepackage{hyperref}
\usepackage{listings}
\title{Software Testing  Lecture 6 \\ Input Domain Modelling}
\author{Justin Pearson}
\date{2020}


%Lecture 4 stuff
\newcommand{\todo}[1]{{\tt ... #1 ... }}
\newcommand{\Reach}{\mathrm{Reach}}
\newcommand{\Path}{\mathrm{path}}

%Lecture 5 stuff
\newcommand{\predset}{{\cal P}}
\newcommand{\clauseset}{{\cal C}}
%Stolen from Stackexchange
%http://tex.stackexchange.com/questions/20609/strikeout-in-math-mode 
\newcommand\hcancel[2][red]{\setbox0=\hbox{$#2$}%
\rlap{\raisebox{.45\ht0}{\textcolor{#1}{\rule{\wd0}{1pt}}}}#2} 

\newcommand{\pauseslide}{\begin{frame}{}
  \begin{center}
    Recording Pause
  \end{center}
  
\end{frame}}
\begin{document}
\lstset{language=C}

\begin{frame}
  \maketitle
\end{frame}
\begin{frame}{Input Domains}
  \begin{itemize}
  \item The {\it input domain} of a program is all the possible inputs to
    that program.
  \item Even for small programs the domain is infinite.
  \item Testing is fundamentally about {\em choosing finite sets} of
    values from the input domain.
  \end{itemize}
  
\end{frame}
\begin{frame}{Input Domains}
  \begin{itemize}
  \item Input parameters define the scope of the input domain:
    \begin{itemize}
    \item Parameters to a method.
    \item Data read from a file.
    \item Global variables.
    \item user level inputs
    \end{itemize}
  \item Domain for each parameter is partitioned into regions.
  \item At least one value is chosen from each region.
  \end{itemize}
  Don't forget the global state can be part of your test case.  
\end{frame}
%%%%%%%%%%%%%%%%%%%%%%%%%%%%%%%%%%%%%%%%
\begin{frame}{Partitions Intuition} 
  \begin{itemize}
  \item Elements in each partition do the same thing; for some value
    of  `the same thing'.
  \item Simple mathematical theory of partitions, allows you to work
    out if you are on the right track.
  \end{itemize}
\end{frame}
%%%%%%%%%%%%%%%%%%%%%%%%%%%%%%%%%%%%%%%%%%%
\begin{frame}{Benefits of Inputs Space Partitions}
  \begin{itemize}
  \item Can be applied at several levels of testing:
    \begin{itemize}
    \item Unit
    \item Integration
    \item System
    \end{itemize}
  \item Relatively easy to apply with no automation.
  \item Easy to adjust to get more of fewer tests.
  \item It is possible to do black box testing, you just need to know
    what the input space is.
  \end{itemize}
    
\end{frame}
%%%%%%%%%%%%%%%%%%%%%%%%%%%%%%%%%%%%%%%%%%
\begin{frame}{Partitions or Equivalence relations} 
  \begin{itemize}
  \item We want to formalise the notion of being the same.
  \item Given a domain. $D$, there are two ways of looking at partitions:
    \begin{itemize}
    \item A partition of $D$ is a set of sets $B_1,\ldots,B_q$  that
      satisfy two properties:
      \begin{itemize}
      \item  Sets must be pairwise disjoint, all $i\neq j$ $B_i \cap
        B_j = \emptyset$.
      \item Sets must cover the whole of $D$, that is $B_1\cup \cdots
        \cup B_q = D$.
      \end{itemize}
    \item A binary relation $\equiv_B$  on $D$ such that:
      \begin{itemize}
      \item for all $d \in D$, $d \equiv_B d$.
      \item for all $d,e \in D$, $d \equiv_B e \Leftrightarrow e
        \equiv_B d$.
      \item For all $d,e,f$ $d \equiv_B e \land e \equiv_B f$ implies
        $d \equiv_B f$.
      \end{itemize}
    \end{itemize}
  \end{itemize}
 \end{frame}
%%%%%%%%%%%%%%%%%%%%%%%%%%%%%%%
\begin{frame}{Partitions or Equivalence relations} 
  \begin{itemize}
  \item Given a partition $B_1,\ldots ,  B_q$ of $D$ then define:
    \[
      d \equiv_B e \Leftrightarrow \exists i. d \in B_i \land e \in B_i
    \]
  \item Given an equivalence relation $\equiv_B$ on $D$ then the
    partition consists of the sets:
    \[
    [d]_{\equiv_B} = \{ e \mid d \equiv_B e \}
    \]  
  \end{itemize}
  
\end{frame}
\begin{frame}{Why is this important?}
  \begin{itemize}
  \item Remember, {\it blocks} should: 
    \begin{itemize}
    \item be disjoint, no overlapping test cases
    \item cover the domain, every element is tested.
    \end{itemize}
  \end{itemize}  
\end{frame}
%%%%%%%%%%%%%%%%%%%%%%%%%%%%%%%%%%%%%%%%%
\begin{frame}{Using Partitions -- Assumptions}
  \begin{itemize}
  \item Choose a value from each partition.
  \item Each value is assumed to be {\em equally useful} for testing.
  \end{itemize}
It is up to the tester to work out what it means to be equally useful.
\end{frame}

\pauseslide

\begin{frame}{Partition design and test design}
  \begin{itemize}
  \item There is no mathematical theorem that says that values in an
    equivalence class act the same way.
  \item When you design a partition as a test designer you are
    deciding want to treat as equal.
  \item Partitions also allow you mathematically define classes of
    tests that you believe all do the same amount of work.
  \end{itemize}
\end{frame}

\begin{frame}[fragile]
 Consider 
  \begin{lstlisting}
     if (x > 0 ) { .... } else { .... }
  \end{lstlisting}
 One of the most important thing about test design is documenting
 what you expect tests to do. Giving a partition is one way of saying
 that you the tester believe that values in equivalence classes behave
 the same.
\end{frame}

\begin{frame}[fragile]{How do we specify partitions?}
  Consider a function
  \begin{lstlisting}
      int f(int x, int y)
    \end{lstlisting}
    \begin{itemize}
    \item The input domain is the Cartesian product of the domains of all the
      input parameters.
    \item In theory you would do the partition on this domain.
    \end{itemize}
    \[
      D = \{ -2147483647 \ldots ... 2147483648\}\times \{ -2147483647
      \ldots ... 2147483648\}
    \]
Sometimes you want to specify partitions on these sorts of domains,
but it is often too complex. 
\end{frame}
%%%%%%%%%%%%%%%%%%%%%%%%%%%%%%%%%%%%%%%%%%%%%%
\begin{frame}{Characteristics}
 Characteristics give a simple way specifying partitions. 
  \begin{itemize}
  \item Application to testing:
    \begin{itemize}
    \item Find {\em characteristics} in the inputs: parameters,
      semantic descriptions.
    \item Partition each characteristic.
    \item Choose tests by combining values from each characteristic.
    \end{itemize}
  \end{itemize}
  Characteristics define blocks.
\end{frame}
%%%%%%%%%%%%%%%%%%%%%%%%%%%%%%%%%%%%%%%%%%%%
\begin{frame}
  \begin{itemize}
  \item Example Characteristics
    \begin{itemize}
    \item Input X is null
    \item Order of an input file (sorted, backwards, arbitary
    \item Minimum separation of two aircrafts.
    \item Input device type (DVD, CD, .... )
    \end{itemize}
  \end{itemize}
\end{frame}
\begin{frame}{Choosing Partitions}
  \begin{itemize}
  \item Choosing (or defining) partitions seems to easy, but it is
    equally easy to get it wrong.
  \item Consider the property the {\it order of the file}.
  \item Define the partition as follows:
    \begin{itemize}
    \item $b_1 = \ $ files sorted in ascending order.
    \item $b_2 = \ $ files sorted in descending order.
    \item $b_3 = \ $ files in an arbitrary order.
    \end{itemize}
  \end{itemize}
  Something is wrong with this. What about files of length $1$.
\end{frame}
\begin{frame}{Choosing Partitions}
  \begin{itemize}
  \item Choosing (or defining) partitions seems to easy, but it is
    equally easy to get it wrong.
  \item Consider the property the order of the file.
  \item Define the partition as follows:
    \begin{itemize}
    \item $b_1 = \ $ files sorted in ascending order.
    \item $b_2 = \ $ files sorted in descending order.
    \item $b_3 = \ $ files in an arbitrary order.
    \end{itemize}
  \end{itemize}
 The blocks are not disjoint. Files of length $1$ belong to
 $b_1$,$b_2$ and $b_3$. Hence for $i\neq j$ $b_i \cap b_j \neq \emptyset$.

\end{frame}
\begin{frame}{Check your Partitions}

  \begin{itemize}
  \item   It is important to double check that you do have a
    partition.
    \item If you  do not have a partition then you have not though
      carefully enough about your test case design.
  \end{itemize}

  
\end{frame}
\begin{frame}{Choosing Partitions}

  \begin{itemize}
  \item You could just be careful.
  \item Or make your partitions binary and combine them. So, we have two
    separate partitions. Generate tests from all possible combinations of
    characteristics. 
    \begin{itemize}
    \item File is sorted ascending:
      \begin{itemize}
      \item $b_1 = \ $ true
      \item $b_2 = \ $ false.
      \end{itemize}
    \item File is sorted descending:
      \begin{itemize}
      \item $b_1 = \ $ true
      \item $b_2 = \ $ false.
      \end{itemize}
    \end{itemize}
    \item Here the combination  is easy. 
  \end{itemize} 
\end{frame}
\begin{frame}{Properties of Partitions}
  \begin{itemize}
  \item Don't forget that partitions should be complete and disjoint;
    if they are not then they have not been considered carefully
    enough.
    \item Review your partitions like any design attempt
    \item Different alternatives should be considered.
  \end{itemize}
\end{frame}
%%%%%%%%%%%%%%%%%%%%%%%%%%%%%%%%%%%%

\pauseslide

\begin{frame}{Modelling the input domain: Five stage plan}

  \begin{itemize}
  \item {\bf Step 1:} Identify testable functions
    \begin{itemize}
    \item Individual methods
    \item Functions
    \item More complicated characteristics, hardware/software,
      etc. You might have to look beyond your code.
    \end{itemize}
  \item {\bf Step 2:} Find all parameters
    \begin{itemize}
    \item Not rocket science. But be complete.
    \item Don't forget global variables.
    \item Don't forget the objects have states. Doing things to an
      empty list is different to doing things to a non-empty list.
    \item Files, databases etc.
    \end{itemize}
  \end{itemize}
  \end{frame}
%%%%%%%%%%%%%%%%%%%%%%%%%%%%%%%%%%%
\begin{frame}{Modelling the input domain: Five stage plan}
  \begin{itemize}
  \item {\bf Step 3:} Model the input domain.
    \begin{itemize}
    \item The domain depends on the parameters.
    \item Define structure in terms of characteristics.
    \item Get blocks and values.
    \item This is where some creativity comes into play, you have to
      decide which characteristics are meaningful to you. 
    \end{itemize}
  \end{itemize}
\end{frame}
%%%%%%%%%%%%%%%%%%%%%%%%%%%%%%%%%%%
\begin{frame}{Modelling the input domain: Five stage plan}
  \begin{itemize}
  \item {\bf Step 4:} {\it Test criteria} for choosing combinations of values.
    \begin{itemize}
    \item See later in the lecture, but you need to combine your
      characteristics.
    \item Choosing all combinations is often infeasible.
    \end{itemize}
  \item {\bf Step 5:} Refine combinations of blocks into test inputs.
    \begin{itemize}
    \item Choose appropriate values from each block.
    \end{itemize}
  \end{itemize}
\end{frame}
%%%%%%%%%%%%%%%%%%%%%%%%%%
\pauseslide
%%%%%%%%%
\begin{frame}{Two Approaches to Input Domain Modelling}
  \begin{enumerate}
  \item {\bf Interface-based} approach.
    \begin{itemize}
    \item Develop characteristics directly from individual input
      parameters.
    \item Just use properties of the domains without considering the
      actual function under test.
    \end{itemize}
  \item {\bf Functionality-based} approach
    \begin{itemize}
    \item Develop characteristics from a behavioural view of the
      program under test.
    \item Use your understanding of what the system should do. 
    \end{itemize}
  \end{enumerate}
  
\end{frame}
%%%%%%%%%%%%%%%%%%%%%%%%%%%%%%%%


\begin{frame}[fragile]{Interface-Based vs Functionality-Based Approach Example}
\begin{verbatim}
  public boolean findElement(List list, Object element)
  //if list or element is null throw NullPointerException
  // else return true if element is in the list.
\end{verbatim}
%%Hello
\begin{itemize}
 \item Interface-Based Approach
   \begin{itemize}
    \item Two parameters: {\tt list}, {\tt element}
    \item Characteristics:
      \begin{itemize}
       \item {\tt list} is null ($b_1$ = true, $b_2$ = false)
       \item {\tt list} is empty ($b_1$ = true, $b_2$ = false)
       \end{itemize}
    \end{itemize}
\end{itemize}
\end{frame}
%

\begin{frame}[fragile]
{Interface-Based vs Functionality-Based Approach Example}
\begin{verbatim}
  public boolean findElement(List list, Object element)
  //if list or element is null throw NullPointerException
  // else return true if element is in the list.
\end{verbatim}
\begin{itemize}
  \item Functionality-Based Approach
   \begin{itemize}
     \item Two parameters: {\tt list}, {\tt element}
     \item Characteristics:
      \begin{itemize}
        \item number of occurrences of {\tt element} in {\tt list}
          \[ (0,1,>1)\]
        \item {\tt element} occurs first in {\tt list}
        \[ (\mathrm{true},\mathrm{false}) \]
      \end{itemize}
    \end{itemize}
%
\end{itemize}
%
\end{frame}
%%
\begin{frame}{Sources for functionality-based characteristics}
  \begin{itemize}
  \item The specification, or intended behaviour.
  \item Pre-conditions: tell what ranges and values you expect for
    input variables.
  \item Post conditions: often give you relationships between input
    and output parameters. 
  \end{itemize}
  

\end{frame}
\begin{frame}{Step 3: Modelling the Input Domain}
  \begin{itemize}
  \item Partitioning characteristics into blocks and values can be
    creative.
  \item More blocks means more tests.
  \item The partitioning often flows directly from the definition of
    the {\em characteristics}.
  \item Strategies for identifying values:
    \begin{itemize}
    \item Include {\em valid}, {\em invalid} and {\em special} values.
    \item Explore {\em boundaries} of domains.
    \item Include values that represent normal use.
    \item Don't forget that partitions must  be {\em complete} {\bf and} {\em disjoint}
    \end{itemize}
  \end{itemize}  
\end{frame}
\begin{frame}{Special Values}
  Some programming languages such as Java have special values such as
  {\tt null}. Use these in your partitions. 
\end{frame}
%%%%%%%%%%%%%%%%%%%%%%%%%%%%%%%%%%%%%%%%%%%%%%%%%%%%%%%%%%%%
\pauseslide
%%%%%%%%%%%%%%%%%%%%%%%%%%%%%%%%%%%%%%%%%%%%%%%%%%%%%%%%%%%% 

\begin{frame}[fragile]{TriType example}
As shown in the book.
  \begin{lstlisting}
  int Tritype(double i, double j, double k)
  \end{lstlisting}
Should return  the type of triangle with side length $i$, $j$ and $k$.
  \begin{itemize}
  \item 3 = scalene (All sides different length)
  \item 2 = equilateral triangle
  \item 1 = isoceles triangle 
  \item 0 = Not a triangle.
  \end{itemize}
It is quite hard to get it right.
\end{frame}
\begin{frame}{TriType Interface-Based Approach}
  \begin{itemize}
  \item One testable function, and three integer inputs.
  \end{itemize}
  \begin{tabular}{||l|l|l|l||}
\hline 
 Characteristic & $b_1$ &  $b_2$ & $b_3$ \\ \hline
 $q_1 = \ $ Relation of i  to $0$ & $>0$ & $=0$ & $<0$ \\
 $q_2 = \ $ Relation of j  to $0$ & $>0$ & $=0$ & $<0$ \\
 $q_3 = \ $ Relation of k  to $0$ & $>0$ & $=0$ & $<0$ \\
\hline    
\end{tabular}
\begin{itemize}
\item Pick one test from each square, a maximum of $3*3*3=27$ tests. 
\item You have to work out which are valid and which are invalid.
\end{itemize}
\end{frame}
\begin{frame}{TriType  Functional Based Appraoch}
  \begin{itemize}
  \item We are classifying triangles.
  \item Geometric approach.
  \end{itemize}
  \begin{tabular}{||l|l|l|l|l||}\hline 
    Characteristic & $b_1$  & $b_2$ & $b_3$ & $b_4$\\
    Geometric  &  scalene & isosceles &
    equilateral & invalid \\
    \hline
  \end{tabular}
  \begin{itemize}
  \item Do we have a partition? 
  \item No, an equilateral triangle is also an isosceles triangle.
  \end{itemize}
\end{frame}
\begin{frame}{TriType  Functional Based Appraoch}
  \begin{tabular}{||l|l|l|l|l||}\hline 
    Characteristic & $b_1$  & $b_2$ & $b_3$ & $b_4$\\
    Geometric  &  scalene & isosceles &
    equilateral & invalid \\
    \hline
  \end{tabular}
  \begin{itemize}
  \item Do we have a partition? 
  \item No, an equilateral is also isosceles.
  \item So refine it to:
  \end{itemize}
{\small
  \begin{tabular}{||l|l|l|l|l||}\hline 
    Characteristic & $b_1$  & $b_2$ & $b_3$ & $b_4$\\
    Geometric  &  scalene & isosceles, not equilateral &
    equilateral & invalid \\
    \hline
  \end{tabular}
}
Example values $(4,5,6) \in b_1$,  $(3,3,4) \in b_2$, $(3,3,3) \in
b_3$, $(3,4,8) \in b_4$.
\end{frame}
%%%%%%%%%%%%%%%%%%%%%%%%%%%%%%%%%%%%%%%%%%%%%%%%%%%%%%%%%%%%%%%%%
\pauseslide
%%%%%%%%%%%%%%%%%%%%%%%%%%%%%%%%%%%%%%%%%%%%%%%%%%%%%%%%%%%%%%%%%
\begin{frame}{Choosing Combinations of Values}
  
  \begin{itemize}
  \item We play the usual game.
  \item Lots of different characteristics, how do we combine them?
  \item Most obvious:
    \begin{itemize}
    \item {\em All Combinations (ACoC)}: All combinations of blocks from all
      characteristics must be used.
    \end{itemize}
  \item Number of tests equals the product of the number of blocks in each
    characteristic. 
  \end{itemize}
  \end{frame}
\begin{frame}{Choosing Combinations of Values}
  \begin{itemize}
  \item Usual story, too many test cases!
  \item We should at least choose a value from each block at least once.
    \begin{itemize}
    \item {\em Each Choice (EC)}: Each value from each block, for each
      characteristic, must be used in at least one test case.
    \end{itemize}
  \item Yields few tests and is cheap, but maybe does not provide such good coverage.
  \end{itemize}
\end{frame}
\begin{frame}{Pair-Wise}
  \begin{itemize}
  \item {\em Pair-Wise (PW)}: A value from each block, for each
    characteristic, must be combined with a value from every other
    block, for each other characteristic.
  \item Given three partitions  $[A,B]$,$[q_1,q_2,q_3]$ and
    $[x,y]$. We would get the requirements:

{\small
\vspace{2mm}
\begin{tabular}{lr}
\vspace{2mm}
  {\large Requirements:} & {\large Combinations:} \\
    \begin{tabular}{lll}
      $(A,q_1)$ & $(B,q_1)$ & $(x,q_1)$ \\
      $(A,q_2)$ & $(B,q_2)$ & $(y,q_1)$ \\
      $(A,q_3)$ & $(B,q_3)$ & $(x,q_2)$ \\
      $(A,x)$  & $(B,x)$   & $(y,q_2)$ \\
      $(A,y)$  & $(B,y)$   & $(x,q_3)$ \\
               &           & $(y,q_3)$
    \end{tabular}
&  
\begin{tabular}{ll}
  $(A,q_1,x)$ & $(B,q_1,y)$ \\
  $(A,q_2,x)$ & $(B,q_2,y)$ \\
  $(A,q_3,y)$ & $(B,q_3,x)$ \\
\end{tabular}
\end{tabular}
}


  \end{itemize}
\end{frame}
\begin{frame}{Constraints Among Characteristics}
  Back to the triangle type.
  \begin{itemize}
  \item Some combinations are infeasible. For example:
    \begin{itemize}
    \item ``Less than zero'' and ``scalene'' are not possible at the same time.
    \end{itemize}
  \item So when you combine tests from different blocks you have to be
    careful.
  \item As you might guess there are lots of possible ways of combining
    things.
    \item For what you are doing, think hard about your partitions and
      combinations. Generate quality test cases.
      
  \end{itemize}

  
\end{frame}
\begin{frame}{Input Space Partitioning Summary}
  \begin{itemize}
  \item Fairly easy to apply
  \item By refining or merging partitions  you can add or remove
    tests.
  \item Various strategies for combining characteristics.
  \item Based only on the input space of the program, not on the
    implementation.
  \item Even if are just looking at the data types and do things like
    ``NULL'', ``Empty'', $\ldots$ you are using partitions.
  \item Designing good partitions can be a good way of generating good
    quality test cases.
    \item Provides the foundation for techniques such as boundary
      value testing\footnote{\tiny{\url{https://www.guru99.com/equivalence-partitioning-boundary-value-analysis.html}}}
  \end{itemize}

\end{frame}
\end{document}

%%% Local Variables:
%%% mode: latex
%%% TeX-master: t
%%% End:
