\documentclass[a4paper]{article}
\usepackage{palatino}
\usepackage{fullpage}
\usepackage[hyphens]{url}
\usepackage{hyperref}
%%%%Draft watermark.
%\usepackage{draftwatermark}
%\SetWatermarkText{DRAFT}
%\SetWatermarkScale{1}
\newcommand{\todo}[1]{{\tt TODO:#1}}
\title{Software Testing 2020}
\author{Justin Pearson }



 
\begin{document}
\maketitle
%adds more characters e.g. \/ for better line breaking     regarding links
\def\UrlBreaks{\do\/\do-} 

\section{Introduction}
The purpose of this document is to make clear how you will be examined, and
what is required of you in the course.  This course essentially consists of
four components:
\begin{enumerate}
\item A series of flipped lectures on aspects of software testing;
    % \item A lab on test driven development (2019-11-20);
  %%   Corona version

\item A TDD exercise that should be done in week 46. Instead of having a
  lab session the lab assistants will be available online that week to
  help with the lab. 
\item Group work on test design;
\item An exam (2021-01-07).
\end{enumerate}
The exam is graded U,3,4,5. The group work and the project is pass or fail.

If you only learn one thing about testing during this course, then you
will fail, but if you learn the following idea ``You should have a
reason for each test. For each test you should be able to explain what
you are trying to test for.'' then you will at least understand why we
study testing. When we examine you, and when you produce tests, we
expect you to be able to explain the reasons behind your tests.


\section{Deliverables and requirements  2020}



\begin{enumerate}
%Precorona
%\item An lab on  test driven development (TDD) for  BibTex.   This is
%  done in groups of 2 or 3. It will be marked at the lab. There is no
%  hand-in. 
%  Post covid
  \item An exercise on  test driven development (TDD) for  author name
    parsing in BibTex. This is to be done individually.
    % In 2019 there are 102 registered, so groups of 5.
    % In 2020 100 registered.
\item Project work on test case design for some Python library. This
  is done in groups of 5 or 6.
\item An exam.

  \end{enumerate}
%exercise -> lab if you have a lab
  Information on the exercise can be found either linked from the
  student portal,  studium  or  via
  \url{http://user.it.uu.se/~justin/Hugo/courses/softwaretesting/}. 

\subsection*{Project Work}

The idea of the project is to pick some python library, and write some
test cases for it. You will have to write both black-box and white-box
tests and tests that provide coverage of selected parts of the
code. The final deliverable is the written report.% and a 10 minute
%presentation.


Using your chosen library, your tasks are as follows:
\begin{list}{$\Rightarrow$}{} 
    \item  Black box testing of the API. You are to produce test cases
      that cover the API.
    \item White box testing. In agreement with your lab assistant you
      are to pick some areas of code in your library to
      cover. Together with the python
      library~\url{https://coverage.readthedocs.io/en/v4.5.x/} you are
      to produce test cases that not only provide statement coverage
      but some sort of path coverage.
    \item For at least one function or method of the library you
      should construct a control flow graph and apply the coverage
      criteria (including prime paths)
      that have been covered in the course to the code. 
    \item You need to document and motivate your design of the test cases in
      a written report. 
\end{list}


\subsubsection*{Schedule for the project}
  \begin{itemize}
  \item By the end of  week 45 you must form a
    group. This will be handled via Studium.  If you are happy to be
    assigned in a random group then email
    \url{it-testing-course@lists.uu.se}. If you do not  join a group
    or do not email me to be assigned a random group then I will
    assume that you are not following the project. 
  \item In week 46 you will meet virtually your assigned lab assistant; before
    the meeting you should pick a python library (see 
    \url{https://pypi.python.org/}). You must agree with your lab
    assistant that you can work on that library. We have to decide if
    the library is too trivial or too complicated. If you are having
    trouble finding a good library to test then you should discuss
    with your lab assistant. 
  \item In week  48 you should have a brief meeting with your assigned lab
    assistant to discuss progress and show a draft report. 

%  \item On 5/12 (week 49) there is a compulsory presentation, where
%    you are to present your library and present some of your test
%    cases and the reasoning behind them. Your presentation is of work
%    in progress.
  \item In week 50 or in the new year, you again meet with your
    lab assistant to discuss your draft of your report. You must
    submit a draft of your report to your assistant before your
    meeting.
    \item Jan 15,  2021 is  the deadline for the final
      report. This should be submitted via Studium. 

\end{itemize}



  \subsubsection*{What should your report contain}
  \begin{itemize}
  \item Description of the python library with examples of use. 
  This should be written (in English) for somebody who has not read the documentation.
  \item Outline of your testing strategy, describing which parts of the library 
    you have performed black box and white box testing on.
  \item A control flow graph for at least one function, with a description of 
    how your tests for this function are related to the graph.
  \item Documentation of all test cases.  When documenting your test cases
    you have to document what your tests are designed to do. This can be done
    in comments in your code, for each test case or group of test cases.
  \end{itemize}
\subsubsection*{Grading criteria for the project}
\begin{itemize}
\item Clarity of presentation. It is important that your document can
  be read without reference to the documentation of the library. This
  means that you will have to re-explain things already in the
  documentation. 
\item Quality of the test cases. What sort of coverage do they
  provide? Do they test enough functionality? Have a range of different 
  techniques been used?
\item Documentation of the test cases. Have you clearly stated what you are
  testing, and why, for each test or group of tests? 

 
\end{itemize}

\subsection*{Unsuitable Python libraries}
This is not an exhaustive list of libraries, just some libraries that
have not worked well in the past.
\begin{itemize}
\item numpy \url{https://pypi.org/project/numpy/}: This has been a
  tempting library for many students in the past, but it is hard to
  come up with good test cases, especially when you are trying to do
  path coverage.
\item Almost anything from the python standard library, especially
  implementations of basic data types.
\item The Python date and time library, is covered by the above item,
  but it is very hard to come up with good test cases.
  
\end{itemize}



\subsection*{Ideas for libraries}
This is not a prescriptive list, but if you are looking for
inspiration some of these libraries have worked quite well in the
past, or they are libraries that we suggest you look at:

\begin{itemize}
\item beautifulsoup \url{https://pypi.org/project/beautifulsoup4/}
  is a web scraping library. This is interesting, because you have to
  work out how to generate and send test cases to the library.
\item PyGreen \url{https://pypi.org/project/pygreen/} PyGreen is a
  static site generator. Any static site generator or a tool to
  convert markdown to some other format. 
\item Panda \url{https://pandas.pydata.org/} is a data analysis tool
  for Python.
\item Diplomacy \url{https://pypi.org/project/diplomacy/} is an
  implementation of the board game Diplomacy. Any similar project will
  be interesting especially if you try to come up with black-box test
  cases that cover the game requirements.
\end{itemize}

Remember that you must discuss the choice of the library with your lab
assistant. The library should not be too simple, nor should be too
complex. You are in Sweden so the library should be lagom.

\end{document}
%%% Local Variables:
%%% mode: latex
%%% TeX-master: t
%%% End:
