\documentclass{paper}
\usepackage{hyperref}
\usepackage{color}
\usepackage{listings}
\lstset{language=Python,
        stringstyle=\ttfamily,
        escapeinside={*@}{@*}}


\title{TDD of  {\tt toEnglish}}
\author{Justin Pearson}

\begin{document}

\maketitle
\section{Introduction}
This is not an original idea. I found the idea on the web in an
article\footnote{\url{http://www.codeproject.com/KB/architecture/TestFirstDevelopment.aspx}}
written by Colin Angus Mackay.

The idea is to write a function that given a number between 0 and 999
inclusive returns a string spelling out the number in English. For
example {\tt toEnglish(43)} should produce the string {\tt 'forty
  three'}. We are going to develop the function in Python. If you are
not familiar with Python, then  pretend it is pseudo code.

\section{Unit Testing in Python}
For full documentation see the manual, but here is enough to get us
started. The code and the test code should be in two different
files. Python, if treated properly, treats files are modules. So your
{\tt test\_to\_English.py} file should import the module containing
your code. We will put the code in a file {\tt toEnglish.py}. To make
life easier put them in the same directory, otherwise tell Python
about file paths.
\begin{lstlisting}
import toEnglish
import unittest

class TesttoEnglish(unittest.TestCase):
    def test_simple(self):
        self.assertEqual(toEnglish.toEnglish(0),'zero')
if __name__ == '__main__':
    unittest.main()
\end{lstlisting}
Two things: we define a class that inherits from the {\tt unittest}
class; and each method name starts with {\tt test}.   The tests are
run by the last two lines. If you execute at the command line {\tt
  python   test\_to\_English.py}, then the test will be run. Again read
the manual to find out more.

\section{Test Driven Development of {\tt toEnglish}}
Remember the TDD mantra:
\begin{description}
\item[Red] Write tests that fail.
\item[Green] Make the tests pass in the most simple way possible.
\item[Refactor]  Is there any duplicated logic in your code that can
  be expressed in more clear and concise way? If so then rewrite. Note
  that sometimes you might have to refactor interfaces. 
\end{description}

\subsection{Red}
The first test:
\begin{lstlisting}
   self.assertEqual(toEnglish.toEnglish(0),'zero')
\end{lstlisting}
We don't yet have a {\tt toEnglish.py} file. We will write the minimal
to get us going.
\begin{lstlisting}
  -------
def toEnglish(n):
    """ Converts a number between 0 and 999 to English """
pass
------- 
\end{lstlisting}
This test will fail:
\begin{verbatim}
=====================================================================
FAIL: test_simple (__main__.TesttoEnglish)
----------------------------------------------------------------------
Traceback (most recent call last):
  File "test_toEnglish.py", line 7, in test_simple
    self.assertEqual(toEnglish.toEnglish(0),'zero')
AssertionError: None != 'zero'

----------------------------------------------------------------------
Ran 1 test in 0.017s

FAILED (failures=1)
\end{verbatim}

\subsection{Green}
So rewrite the code to pass the test. If we are being religious about
this, then we only write the minimal code for it to pass.

\begin{lstlisting}
  def toEnglish(n):
    """ Converts a number between 0 and 999 to English """

    return('return zero')
\end{lstlisting}
Nothing to refactor. 
\subsection{Red}
Lets write some more tests to make it fail.
\begin{lstlisting}
     self.assertEqual(toEnglish.toEnglish(1),'one')
\end{lstlisting}
The code will not pass the test:
\begin{verbatim}
    self.assertEqual(toEnglish.toEnglish(1),'one')
AssertionError: 'zero' != 'one'
\end{verbatim}
\subsection{Green}
\begin{lstlisting}
  def toEnglish(n):
    """ Converts a number between 0 and 999 to English """
    if   n==0:
        return('zero')
    elif n==1:
        return('one')
\end{lstlisting}
Now the code will pass the tests. We might be tempted to make bigger
steps, but while we are still trying to understand the problem, we only
take the smallest steps.
\subsection{Red}
\begin{lstlisting}
          self.assertEqual(toEnglish.toEnglish(2),'two')
\end{lstlisting}
Fails.
\begin{verbatim}
  File "test_toEnglish.py", line 9, in test_simple
    self.assertEqual(toEnglish.toEnglish(2),'two')
AssertionError: None != 'two'
\end{verbatim}
\subsection{Green}
\begin{lstlisting}
def toEnglish(n):
    """ Converts a number between 0 and 999 to English """
    if   n==0:
        return('zero')
    elif n==1:
        return('one')
    elif n==2:
        return('two')
\end{lstlisting}
\subsection{Red}
We are on a roll. For the moment we know what happens we add the tests
and extend the {\tt if} statement.
\begin{lstlisting}
    self.assertEqual(toEnglish.toEnglish(3),'three')
    self.assertEqual(toEnglish.toEnglish(4),'four')
    self.assertEqual(toEnglish.toEnglish(5),'five')
    self.assertEqual(toEnglish.toEnglish(6),'six')
    self.assertEqual(toEnglish.toEnglish(7),'seven')
    self.assertEqual(toEnglish.toEnglish(8),'eight')
    self.assertEqual(toEnglish.toEnglish(9),'nine')
    self.assertEqual(toEnglish.toEnglish(10),'ten')
    self.assertEqual(toEnglish.toEnglish(11),'eleven')
\end{lstlisting}
\subsection{Green}
\begin{lstlisting}
  def toEnglish(n):
    """ Converts a number between 0 and 999 to English """
    if   n==0:
        return('zero')
    elif n==1:
        return('one')
    elif n==2:
        return('two')
    elif n==3:
        return('three')
    elif n==4:
        return('four')
    elif n==5:
        return('five')
    elif n==6:
        return('six')
    elif n==7:
        return('seven')
    elif n==8:
        return('eight')
    elif n==9:
        return('nine')
    elif n==10:
        return('ten')
    elif n==11:
        return('eleven')
\end{lstlisting}
The {\tt elif} is a Python construct for {\tt else if}.

\subsection{Refactor}
Why don't we refactor? Use a list instead of  a bunch of {\tt if} and
 {\tt  elif} statements. 
\begin{lstlisting}
  def toEnglish(n):
    """ Converts a number between 0 and 999 to English """
    numbers = ['zero','one','two','three','four','five','six',
               'seven','eight','nine','ten','eleven']
    return(numbers[n])
\end{lstlisting}
Before we proceed we check that all the tests are passed, and they
are.
\subsection{Red}
Again we are taking slightly bigger steps, but we know what is going
on.
\begin{lstlisting}
        self.assertEqual(toEnglish.toEnglish(12),'twelve')	
        self.assertEqual(toEnglish.toEnglish(13),'thirteen')	
        self.assertEqual(toEnglish.toEnglish(14),'fourteen')
        self.assertEqual(toEnglish.toEnglish(15),'fifteen')
        self.assertEqual(toEnglish.toEnglish(16),'sixteen')
        self.assertEqual(toEnglish.toEnglish(17),'seventeen')
        self.assertEqual(toEnglish.toEnglish(18),'eighteen')
        self.assertEqual(toEnglish.toEnglish(19),'nineteen')
        self.assertEqual(toEnglish.toEnglish(20),'twenty')
        self.assertEqual(toEnglish.toEnglish(21),'twenty one')
\end{lstlisting}
\subsection{Green}
\begin{lstlisting}
  def toEnglish(n):
    """ Converts a number between 0 and 999 to English """
    numbers = ['zero','one','two','three','four','five','six',
               'seven','eight','nine','ten','eleven','twelve',
               'thirteen','fourteen','fifteen','sixteen',
               'seventeen','eighteen','nineteen','twenty','twenty one']
    return(numbers[n])
\end{lstlisting}
\subsection{Refactor}
Time to refactor {\tt 'twenty one'} 
 is {\tt 'twenty' + ' ' + 'one'}. This gives us the  following code:
 \begin{lstlisting}
   def toEnglish(n):
    """ Converts a number between 0 and 999 to English """
    numbers = ['zero','one','two','three','four','five','six',
               'seven','eight','nine','ten','eleven','twelve',
               'thirteen','fourteen','fifteen','sixteen',
               'seventeen','eighteen','nineteen','twenty','twenty one']
    if n in range(0,20):
        return(numbers[n])
    else:
        return('twenty' + ' ' + numbers[n-20])
 \end{lstlisting}
Do not forget to rerun all your tests when you refactor. Well when we
do, we get the error
\begin{verbatim}
  File "test_toEnglish.py", line 27, in test_simple
    self.assertEqual(toEnglish.toEnglish(20),'twenty')
AssertionError: 'twenty zero' != 'twenty'
\end{verbatim}

Studying the code gives us an error. I had assumed that {\tt
  range(0,20)} produced all the numbers up to and including twenty. It
only goes up to 19. It is in the manual, but I found out by
testing. So this gives us a new version
\begin{lstlisting}
  def toEnglish(n):
    """ Converts a number between 0 and 999 to English """
    numbers = ['zero','one','two','three','four','five','six',
               'seven','eight','nine','ten','eleven','twelve',
               'thirteen','fourteen','fifteen','sixteen',
               'seventeen','eighteen','nineteen','twenty','twenty one']
    if n in range(0,21):
        return(numbers[n])
    else:
        return('twenty' + ' ' + numbers[n-20])
\end{lstlisting}
Rerun the tests and every thing is well.

\subsection{Red}
Well now that we have a feel for our problem we can start generating
the test cases in larger steps.
\begin{lstlisting}
        self.assertEqual(toEnglish.toEnglish(22),'twenty two')
        self.assertEqual(toEnglish.toEnglish(23),'twenty three')
        self.assertEqual(toEnglish.toEnglish(29),'twenty nine')
\end{lstlisting}
Well we still are not at our red state because all our tests are
passed.
\begin{lstlisting}
       self.assertEqual(toEnglish.toEnglish(30),'thirty')
\end{lstlisting}
Now we fail.
\begin{verbatim}
  File "test_toEnglish.py", line 32, in test_simple
    self.assertEqual(toEnglish.toEnglish(30),'thirty')
AssertionError: 'twenty ten' != 'thirty'
\end{verbatim}
\subsection{Green}
So thirty should  treated as a special case.
\begin{lstlisting}
  def toEnglish(n):
    """ Converts a number between 0 and 999 to English """
    numbers = ['zero','one','two','three','four','five','six',
               'seven','eight','nine','ten','eleven','twelve',
               'thirteen','fourteen','fifteen','sixteen',
               'seventeen','eighteen','nineteen']
    if n in range(0,21):
        return(numbers[n])
    elif n in range(21,30):
        return('twenty' + ' ' + numbers[n-20])
    elif n == 30 :
        return('thirty')
    elif n in range(31,40):
        return('thirty' + ' ' + numbers[n-30])
\end{lstlisting}
\subsection{Red}
All is still going well tests are passing. 
\begin{lstlisting}
        self.assertEqual(toEnglish.toEnglish(32),'thirty two')
        self.assertEqual(toEnglish.toEnglish(39),'thirty nine')
\end{lstlisting}
Our goal is to find tests that fail. Don't develop any code until you
can find a test that fails.
\begin{lstlisting}
        self.assertEqual(toEnglish.toEnglish(40),'forty')
        self.assertEqual(toEnglish.toEnglish(41),'forty one')
        self.assertEqual(toEnglish.toEnglish(49),'forty nine')
\end{lstlisting}
\subsection{Green}
\begin{lstlisting}
  def toEnglish(n):
    """ Converts a number between 0 and 999 to English """
    numbers = ['zero','one','two','three','four','five','six',
               'seven','eight','nine','ten','eleven','twelve',
               'thirteen','fourteen','fifteen','sixteen',
               'seventeen','eighteen','nineteen','twenty','twenty one']
    if n in range(0,21):
        return(numbers[n])
    elif n in range(21,30):
        return('twenty' + ' ' + numbers[n-20])
    elif n == 30 :
        return('thirty')
    elif n in range(31,40):
        return('thirty' + ' ' + numbers[n-30])
    elif n == 40:
        return('forty')
    elif n in range(41,50):
        return('forty' + ' ' + numbers[n-40])
\end{lstlisting}
\subsection{Refactor}
Now time to refactor. Looking at the code we can use a similar trick
with a list instead of all those {\tt if} statements.
\begin{lstlisting}
  def toEnglish(n):
    """ Converts a number between 0 and 999 to English """
    numbers = ['zero','one','two','three','four','five','six',
               'seven','eight','nine','ten','eleven','twelve',
               'thirteen','fourteen','fifteen','sixteen',
               'seventeen','eighteen','nineteen']
    tens = ['twenty','thirty','forty','fifty','sixty',
            'seventy','eighty','ninety']
    numberOfTens  = n // 10
    numberOfUnits = n % 10
#We always treat the numbers 0-19 as special numbers.
    if n in range(0,20):
        return(numbers[n])
    return_string = tens[numberOfTens]
    if numberOfUnits > 0 :
        return_string = return_string + ' ' + numbers[numberOfUnits]
    return(return_string)
\end{lstlisting}
Run the tests we have to see if we didn't mess anything up.
\begin{verbatim}
  File "test_toEnglish.py", line 27, in test_simple
    self.assertEqual(toEnglish.toEnglish(20),'twenty')
AssertionError: 'forty' != 'twenty'
\end{verbatim}
Problem with the {\tt tens} list. I got the indexing wrong. One way to
fix it is to add two dummy values at the head of the list.
\begin{lstlisting}
  def toEnglish(n):
    """ Converts a number between 0 and 999 to English """
    numbers = ['zero','one','two','three','four','five','six',
               'seven','eight','nine','ten','eleven','twelve',
               'thirteen','fourteen','fifteen','sixteen',
               'seventeen','eighteen','nineteen']
    tens = ['','','twenty','thirty','forty','fifty','sixty',
            'seventy','eighty','ninety']
    numberOfTens  = n // 10
    numberOfUnits = n % 10
#We always treat the numbers 0-19 as special numbers.
    if n in range(0,20):
        return(numbers[n])
    return_string = tens[numberOfTens]
    if numberOfUnits > 0 :
        return_string = return_string + ' ' + numbers[numberOfUnits]
    return(return_string)
\end{lstlisting}
Now run the tests and all is well.
\subsection{Red}
When adding more tests we don't need to add one for every value between
zero and ninety nine, but we need to check border cases. We are still
looking for test that make our code fail so we can write some code.
\begin{lstlisting}
        self.assertEqual(toEnglish.toEnglish(100),'one hundred')
\end{lstlisting}
\subsection{Green}
So we rewrite the code with the minimal effort to pass the test (it
was late in the day).
\begin{lstlisting}
  def toEnglish(n):
    """ Converts a number between 0 and 999 to English """
    numbers = ['zero','one','two','three','four','five','six',
               'seven','eight','nine','ten','eleven','twelve',
               'thirteen','fourteen','fifteen','sixteen',
               'seventeen','eighteen','nineteen']
    tens = ['','','twenty','thirty','forty','fifty','sixty',
            'seventy','eighty','ninety']
    numberOfTens  = n // 10
    numberOfUnits = n % 10
#We always treat the numbers 0-19 as special numbers.
    if n in range(0,20):
        return(numbers[n])
    elif numberOfTens in range(1,10):
        return_string = tens[numberOfTens]
        if numberOfUnits > 0 :
            return_string = return_string + ' ' + numbers[numberOfUnits]
        return(return_string)
    return('one hundred')
\end{lstlisting}
\subsection{Red}
If we add the test
\begin{lstlisting}
       self.assertEqual(toEnglish.toEnglish(200),'two hundred')
\end{lstlisting}
then the code will fail and we will have to rewrite something.
\begin{lstlisting}
  def toEnglish(n):
    """ Converts a number between 0 and 999 to English """
    numbers = ['zero','one','two','three','four','five','six',
               'seven','eight','nine','ten','eleven','twelve',
               'thirteen','fourteen','fifteen','sixteen',
               'seventeen','eighteen','nineteen']
    tens = ['','','twenty','thirty','forty','fifty','sixty',
            'seventy','eighty','ninety']
    numberOfHundreds = n // 100
    numberOfTens  = n // 10
    numberOfUnits = n % 10
#We always treat the numbers 0-19 as special numbers.
    return_string = ''
    if n in range(0,20):
        return(numbers[n])
    elif numberOfHundreds in range(1,10):
        return_string = return_string +\
                        numbers[numberOfHundreds] + ' hundred'
    elif numberOfTens in range(1,10):
        return_string = return_string + tens[numberOfTens]
        if numberOfUnits > 0 :
            return_string = return_string + ' ' + numbers[numberOfUnits]
    return(return_string)
\end{lstlisting}
\subsection{Red}
We add the test
\begin{lstlisting}
   self.assertEqual(toEnglish.toEnglish(201),'two hundred and one')
\end{lstlisting}
Of course the test fails. 
\subsection{Green/Refactor}
Part of the problem is our numberOfTens calculation assumes that n is
two digits and we got our logic wrong with the elif stuff. But we can
do some refactoring and use toEnglish ourselves using the fact that:
\begin{lstlisting}
 'k*100 + w' = toEnglish(k) + 'hundred and ' + toEnglish(w)  
\end{lstlisting}

We are cheating a bit by doing the refactoring in one step, but by
doing all these tests we have a deeper understanding of the problem.

So this gives us a new attempt:
\begin{lstlisting}
def toEnglish(n):
    """ Converts a number between 0 and 999 to English """
    numbers = ['zero','one','two','three','four','five','six',
               'seven','eight','nine','ten','eleven','twelve',
               'thirteen','fourteen','fifteen','sixteen',
               'seventeen','eighteen','nineteen']
    tens = ['','','twenty','thirty','forty','fifty','sixty',
            'seventy','eighty','ninety']
    numberOfHundreds = n // 100
    numberOfTens  = n // 10
    numberOfUnits = n % 10
#We always treat the numbers 0-19 as special numbers.
    return_string = ''
    if n in range(0,20):
        return(numbers[n])
    elif numberOfHundreds in range(1,10):
        return_string = return_string +\
                        numbers[numberOfHundreds] + ' hundred'
        n = n - numberOfHundreds*100
        return_string = return_string + ' and ' + toEnglish(n)
        return(return_string)
    elif numberOfTens in range(1,10):
        return_string = return_string + tens[numberOfTens]
        if numberOfUnits > 0 :
            return_string = return_string + ' ' + numbers[numberOfUnits]
            return(return_string)
  \end{lstlisting}
Lets run the tests.
\begin{verbatim}

AssertionError: 'one hundred and zero' != 'one hundred'


AssertionError: None != 'twenty'

AssertionError: None != 'thirty'
\end{verbatim}
We seem to have messed up something with the numbers less than one
hundred, but lets just fix the problems one at a time. We'll fix the
{\tt 'one hundred and zero'} problem.
\begin{lstlisting}
  def toEnglish(n):
    """ Converts a number between 0 and 999 to English """
    numbers = ['zero','one','two','three','four','five','six',
               'seven','eight','nine','ten','eleven','twelve',
               'thirteen','fourteen','fifteen','sixteen',
               'seventeen','eighteen','nineteen']
    tens = ['','','twenty','thirty','forty','fifty','sixty',
            'seventy','eighty','ninety']
    numberOfHundreds = n // 100
    numberOfTens  = n // 10
    numberOfUnits = n % 10
#We always treat the numbers 0-19 as special numbers.
    return_string = ''
    if n in range(0,20):
        return(numbers[n])
    elif numberOfHundreds in range(1,10):
        return_string = return_string +\
                        numbers[numberOfHundreds] + ' hundred'
        n = n - numberOfHundreds*100
        if n>0:
            return_string = return_string + ' and ' + toEnglish(n)
        return(return_string)
    elif numberOfTens in range(1,10):
        return_string = return_string + tens[numberOfTens]
        if numberOfUnits > 0 :
            return_string = return_string + ' ' + numbers[numberOfUnits]
            return(return_string)
\end{lstlisting}
But we still get the problems with 
\begin{verbatim}
AssertionError: None != 'twenty'

AssertionError: None != 'thirty'
\end{verbatim}
It means that I've got some of the if then else logic wrong; an easy
problem in Python. Good job we have got tests to see if we have messed
things up.

So finally some code that passes all the tests
\begin{lstlisting}
  def toEnglish(n):
    """ Converts a number between 0 and 999 to English """
    numbers = ['zero','one','two','three','four','five','six',
               'seven','eight','nine','ten','eleven','twelve',
               'thirteen','fourteen','fifteen','sixteen',
               'seventeen','eighteen','nineteen']
    tens = ['','','twenty','thirty','forty','fifty','sixty',
            'seventy','eighty','ninety']
    numberOfHundreds = n // 100
    numberOfTens  = n // 10
    numberOfUnits = n % 10
#We always treat the numbers 0-19 as special numbers.
    return_string = ''
    if n in range(0,20):
        return(numbers[n])
    if numberOfHundreds in range(1,10):
        return_string = return_string +\
                         numbers[numberOfHundreds] + ' hundred'
        n = n - numberOfHundreds*100
        if n>0:
            return_string = return_string + ' and ' + toEnglish(n)
        return(return_string)
    if numberOfTens in range(1,10):
        return_string = return_string + tens[numberOfTens]
        if numberOfUnits > 0 :
            return_string = return_string + ' ' + numbers[numberOfUnits]
        return(return_string)
\end{lstlisting}
The code could be improved. The logic it not quite crystal clear. We
have multiple return points and it not clear that the function will
always return something for a number 0-999. Exercise refactor and
retest.

\section{All our test cases}
\begin{lstlisting}
import toEnglish
import unittest

class TesttoEnglish(unittest.TestCase):

    def test_simple(self):
        self.assertEqual(toEnglish.toEnglish(0),'zero')
        self.assertEqual(toEnglish.toEnglish(1),'one')
        self.assertEqual(toEnglish.toEnglish(2),'two')
        self.assertEqual(toEnglish.toEnglish(3),'three')
        self.assertEqual(toEnglish.toEnglish(4),'four')
        self.assertEqual(toEnglish.toEnglish(5),'five')
        self.assertEqual(toEnglish.toEnglish(6),'six')
        self.assertEqual(toEnglish.toEnglish(7),'seven')
        self.assertEqual(toEnglish.toEnglish(8),'eight')
        self.assertEqual(toEnglish.toEnglish(9),'nine')
        self.assertEqual(toEnglish.toEnglish(10),'ten')
        self.assertEqual(toEnglish.toEnglish(11),'eleven')
        self.assertEqual(toEnglish.toEnglish(12),'twelve')
        self.assertEqual(toEnglish.toEnglish(13),'thirteen')
        self.assertEqual(toEnglish.toEnglish(14),'fourteen')
        self.assertEqual(toEnglish.toEnglish(15),'fifteen')
        self.assertEqual(toEnglish.toEnglish(16),'sixteen')
        self.assertEqual(toEnglish.toEnglish(17),'seventeen')
        self.assertEqual(toEnglish.toEnglish(18),'eighteen')
        self.assertEqual(toEnglish.toEnglish(19),'nineteen')
        self.assertEqual(toEnglish.toEnglish(20),'twenty')
    def test_tens(self):
        self.assertEqual(toEnglish.toEnglish(21),'twenty one')
        self.assertEqual(toEnglish.toEnglish(22),'twenty two')
        self.assertEqual(toEnglish.toEnglish(23),'twenty three')
        self.assertEqual(toEnglish.toEnglish(29),'twenty nine')
        self.assertEqual(toEnglish.toEnglish(30),'thirty')
        self.assertEqual(toEnglish.toEnglish(31),'thirty one')
        self.assertEqual(toEnglish.toEnglish(32),'thirty two')
        self.assertEqual(toEnglish.toEnglish(39),'thirty nine')
        self.assertEqual(toEnglish.toEnglish(40),'forty')
        self.assertEqual(toEnglish.toEnglish(41),'forty one')
        self.assertEqual(toEnglish.toEnglish(49),'forty nine')
        self.assertEqual(toEnglish.toEnglish(50),'fifty')
        self.assertEqual(toEnglish.toEnglish(51),'fifty one')
        self.assertEqual(toEnglish.toEnglish(59),'fifty nine')
        self.assertEqual(toEnglish.toEnglish(60),'sixty')
        self.assertEqual(toEnglish.toEnglish(61),'sixty one')
        self.assertEqual(toEnglish.toEnglish(69),'sixty nine')
        self.assertEqual(toEnglish.toEnglish(70),'seventy')
        self.assertEqual(toEnglish.toEnglish(71),'seventy one')
        self.assertEqual(toEnglish.toEnglish(79),'seventy nine')
        self.assertEqual(toEnglish.toEnglish(80),'eighty')
        self.assertEqual(toEnglish.toEnglish(81),'eighty one')
        self.assertEqual(toEnglish.toEnglish(85),'eighty five')
        self.assertEqual(toEnglish.toEnglish(90),'ninety')
        self.assertEqual(toEnglish.toEnglish(91),'ninety one')
        self.assertEqual(toEnglish.toEnglish(99),'ninety nine')
    def test_hundreds(self):
        self.assertEqual(toEnglish.toEnglish(100),'one hundred')
        self.assertEqual(toEnglish.toEnglish(200),'two hundred')
        self.assertEqual(toEnglish.toEnglish(201),'two hundred and one')
        
      

if __name__ == '__main__':
    unittest.main()


\end{lstlisting}
\end{document}
